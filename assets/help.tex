% !TEX TS-program = pdflatex
% !TEX encoding = UTF-8 Unicode

% This is a simple template for a LaTeX document using the "article" class.
% See "book", "report", "letter" for other types of document.

\documentclass[11pt]{article} % use larger type; default would be 10pt

\usepackage[utf8]{inputenc} % set input encoding (not needed with XeLaTeX)

\usepackage[UKenglish]{babel}

%%% Examples of Article customizations
% These packages are optional, depending whether you want the features they provide.
% See the LaTeX Companion or other references for full information.

%%% PAGE DIMENSIONS
\usepackage{geometry} % to change the page dimensions
\geometry{a4paper} % or letterpaper (US) or a5paper or....
% \geometry{margin=2in} % for example, change the margins to 2 inches all round
% \geometry{landscape} % set up the page for landscape
%   read geometry.pdf for detailed page layout information

\usepackage{graphicx} % support the \includegraphics command and options

\usepackage[parfill]{parskip} % Activate to begin paragraphs with an empty line rather than an indent

%%% PACKAGES
\usepackage{booktabs} % for much better looking tables
\usepackage{array} % for better arrays (eg matrices) in maths
\usepackage{paralist} % very flexible & customisable lists (eg. enumerate/itemize, etc.)
\usepackage{verbatim} % adds environment for commenting out blocks of text & for better verbatim
\usepackage{subfig} % make it possible to include more than one captioned figure/table in a single float
\usepackage{hyperref}
% These packages are all incorporated in the memoir class to one degree or another...

%%% HEADERS & FOOTERS
\usepackage{fancyhdr} % This should be set AFTER setting up the page geometry
\pagestyle{fancy} % options: empty , plain , fancy
\renewcommand{\headrulewidth}{0pt} % customise the layout...
\lhead{}\chead{}\rhead{}
\lfoot{}\cfoot{\thepage}\rfoot{}

%%% SECTION TITLE APPEARANCE
\usepackage{sectsty}
\allsectionsfont{\sffamily\mdseries\upshape} % (See the fntguide.pdf for font help)
% (This matches ConTeXt defaults)

%%% ToC (table of contents) APPEARANCE
\usepackage[nottoc,notlof,notlot]{tocbibind} % Put the bibliography in the ToC
\usepackage[titles,subfigure]{tocloft} % Alter the style of the Table of Contents
\renewcommand{\cftsecfont}{\rmfamily\mdseries\upshape}
\renewcommand{\cftsecpagefont}{\rmfamily\mdseries\upshape} % No bold!

%%% END Article customizations

%%% The "real" document content comes below...

\title{Belldroid Documentation}
\author{Chris Rees}
%\date{} % Activate to display a given date or no date (if empty),
         % otherwise the current date is printed 

\newcommand{\app}{\emph{Belldroid} }
\newcommand{\setting}{\texttt}

\begin{document}
\maketitle
\thanks{\href{https://code.google.com/p/belldroid/}{https://code.google.com/p/belldroid/} \\ \href{http://www.bayofrum.net/belldroid/doc.pdf}{http://www.bayofrum.net/belldroid/doc.pdf}}
\section*{Disclaimer}

\begin{quotation}
\app is a tool for bellringers to practice specific skills as detailed below.
Some understanding of change ringing is needed, and you will not learn the basics from this tool.
Please feel free to enjoy it, but you must at least understand either how call changes work, or the very basics of method ringing.
\end{quotation}

\section*{Introduction}

\app is a simulator for English-style church bell change ringing.  There are several modes, described here.

\tableofcontents

\newpage

\section{Call changes}

To use this function, go into \setting{Settings}, choose the \setting{Number of bells}, and from \setting{My Bell} choose \setting{None-- call changes}.

At the bottom of \setting{Settings} there is an option for calling ``up'' or ``down''.
Choose your preferred calling style by toggling it.
Once you have started ringing, you can touch any bell you wish to call out of position, and the call will be shown above.
For example, to move the \setting{Two} into lead, with \setting{Calling style} set to ``up'', you would touch the \setting{Treble}, otherwise touch the \setting{Two}.

\section{Striking errors}

To use this function, go into \setting{Settings}, choose your \setting{Number of bells}, and from \setting{My Bell} select \setting{Choose the wrongly striking bell}.

\setting{Striking variance} will activate, and you can choose whether to make the badly struck bell easy to detect (large error), or hard to detect (small error).
Using this function is simply a case of getting the bells ringing (\setting{Look to!}), and then touching the bell that you consider to be striking wrongly (too close or too wide).
Feedback is provided, and if you choose correctly, another bell starts to make errors.

\section{Practice methods}

This function can actually be combined with either of the above, although the nature depends on \setting{My Bell}.

\begin{description}

\item[1 - 12] Take control of the named bell, and it rings only when you touch it.  Accuracy of your striking is scored.

\item[None-- call changes] Listen to methods after you have called bells out of place.

\item[Choose the wrongly striking bell] Pick badly striking bells out during a method.

\end{description}

Use \setting{Update method database} to allow choice of methods from those available at methods.ringing.org, and choose a method to practise.
Alternatively, you can just use it to listen to the methods!

\end{document}
